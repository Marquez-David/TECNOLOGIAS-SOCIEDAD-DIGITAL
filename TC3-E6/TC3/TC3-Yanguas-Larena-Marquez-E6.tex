\input{../Config/preamble-anteproyecto.tex} 
\input{../Config/myconfig.tex}
\input{../Config/postamble.tex} 

\graphicspath{{../Book/figures/}}


\title{Telefonía móvil 6G}
\date{3 de Febrero de 2021}            
\author{Tecnologías para la Sociedad Digital}

\begin{document}

\input{../Config/setlanguagedependentissues.tex}
\maketitle
\begin{description}                             
  \ifthenelse{\equal{\mybooklanguage}{english}}   
  {        
  \item[Título en inglés:] \mybooktitleenglish   
  }                                            
  {                                           
  }                                              
\item[Equipo:] E6   
\item[Autores:] Carlos Yanguas Durán, Alberto Larena Luengo, David Márquez Mínguez                
\item[\expandafter\makefirstuc\expandafter{\mybookTutorOrTutores}:] Jesús Alpuente Hermosilla, Juan Antonio Martínez Rojas          
\end{description}   

\begin{center}
 {\bfseries \Large Resumen}
\end{center}
Con la actual popularidad de los dispositivos móviles nos aparece la duda de cómo evolucionará este ámbito. Mientras actualmente se está integrando el 5G en todo el mundo, ya se ha comenzado a abstraer lo que vendrá después y será el 6G. El 6G producirá una transformación sin precedentes generando una revolución tecnológica similar a la que está produciendo actualmente el 5G. La revolución de la telefónia móvil 6G destacará en la inteligencia artificial ubicua y en servicios de red, resolviendo los problemas que vayan apareciendo durante su implementación.                            

\section{Introducción}
\label{sec:introduccion}

No ha terminado de implantarse la tecnología 5G en nuestras  vidas y comenzamos a hablar del 6G, tecnología que promete ser la evolución de las cosas conectadas a la inteligencia conectada. Entre sus servicios más llamativos e innovadores destaca la inteligencia artificial implantada en dispositivos inalámbricos, capacidad de lograr tasas de transferencia de hasta 1Tb/s, posibilidad de soportar dispositivos IoT sin necesidad de batería y cobertura de red global con banda ancha omnipresente. Sin embargo, todos estos avances requieren de solventar una serie de requerimientos que parecen tan inalcanzables como los servicios que puede llegar a ofrecer esta misma tecnología. 

Entre los más llamativos podemos hablar de soporte de almacenamiento de contenidos en cache, sustitución de los tradicionales algoritmos matemáticos por innovadores algoritmos de inteligencia artificial, entrenamiento de la red de manera distribuida entre los diferentes dispositivos y no en la nube. Además, esta tecnología también deberá de ofrecer plataformas flexibles para soportar la mezcla heterogénea entre la nube, el borde y los dispositivos finales.

A continuación, analizaremos si el 6G es una tecnología de ciencia ficción, o si por el contrario estos inconvenientes tienen soluciones ya planteadas y podría llegar a nuestras vidas antes de lo imaginado.

\section{Discusión}
\label{sec:discusion}
Comencemos analizando como podrían llegar a implantarse estos servicios tan revolucionarios que pretende ofrecer la tecnología 6G:
\begin{enumerate}
\item {\bfseries Inteligencia artificial implantada en dispositivos inalámbricos:  }
Uno de los principales problemas de este servicio, es la necesidad de grandes recursos computacionales para el entrenamiento de la red, este primer problema podría ser resuelto mediante entrenamiento distribuido, ya que los dispositivos móviles cada vez tienen una mayor capacidad y disponen de una gran cantidad de recursos para llevar esta función a cabo. Por tanto, parece viable la posibilidad de realizar un entrenamiento distribuido mediante el procesamiento de datos a nivel local por parte de los diferentes dispositivos conectados a la red.

El principal problema de esta solución es que desemboca en un segundo problema, que es la transmisión de los datos resultantes de este entrenamiento, ya que las bandas wifi suponen un gran cuello de botella durante esta parte del algoritmo para implementar inteligencia artificial. 

La solución a este segundo inconveniente será la explotación de la propiedad de superposición de los canales de acceso múltiple inalámbrico, que consiste en la selección del mayor número de dispositivos para realizar esta transmisión de datos, distribuyendo el peso del cómputo del entrenamiento entre el número máximo de dispositivos, y permitiendo además agilizar la transferencia de los datos resultantes de este entrenamiento.

Por tanto, este primer servicio parece alcanzable, sin embargo, ¿Qué ocurriría si un gran número de dispositivos inalámbricos migrasen de repente a otra red?. El comportamiento de esta red que recibe esta cantidad de ¨inmigrantes tecnológicos¨, podría verse afectado por el diferente comportamiento de estos individuos, por ejemplo, haciéndole predecir a la red que a una determinada hora se van a demandar muchos más recursos de los que realmente se van a necesitar, dejando sin los mismos a otras redes que si que necesitarán estos.

\item {\bfseries Transferencia de datos con velocidades de transmisión de hasta 1Tb/s:  }
Según este artículo, la llegada del 6G aumentará la velocidad de transferencia de datos pero, ¿En que nos afectará a nosotros y qué inconvenientes genera la implementación dicho sistema?. Al aumentar la velocidad de la transferencia de datos, nos encontraremos con tarifas de internet más elevadas, haciendo que las tarifas de menor velocidad queden obsoletas y teniendo que contratar tarifas a con mayor velocidad y precio debido a que los proveedores de servicios de internet aumentarían sus gastos.

El aumento del gasto de los proveedores sería por la necesidad de operar con frecuencias más elevadas a una mayor escala, el aumento de antenas para evitar la pérdida de datos y un aumento del consumo de energía debido a la necesidad de hardware adicional. Este servicio parece alcanzable simplemente aumentando el número de dispositivos, aunque nos traería un mayor coste.
\item {\bfseries Conexión de dispositivos Iot a la red sin necesidad de batería:  }
Este servicio suena muy interesante debido a que en la actualidad siempre debemos estar pendientes de que un dispositivo electrónico tenga carga de batería o esté conectado a un enchufe. Esto se podría lograr mediante el uso de energía renovable como la implementación de un panel solar en dicho dispositivo.
\item {\bfseries Cobertura omnipresente de banda ancha:  }
A medida que transcurre el tiempo, va aumentando la conectividad en todo el mundo debido al aumento de recursos en dicho ámbito. Estamos seguros de que algún día la cobertura será omnipresente conllevando el aumento de gasto energético y pudiendo provocar un aumento de basura espacial debido al aumento de satélites.
\end{enumerate}

\section{Conclusiones}
\label{sec:conclusiones}
La telefonía móvil 6G es algo que veremos en un no tan lejano futuro y revolucionará la tecnología tal y como la conocemos aumentando la velocidad y conectividad a internet y resolviendo los problemas que vayan apareciendo en su camino.\nocite{6G}

\input{../Book/biblio/bibliography.tex}          



\end{document}

