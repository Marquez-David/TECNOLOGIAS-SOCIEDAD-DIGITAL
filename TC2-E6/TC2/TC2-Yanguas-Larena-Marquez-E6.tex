\input{../Config/preamble-anteproyecto.tex} 
\input{../Config/myconfig.tex}
\input{../Config/postamble.tex} 

\graphicspath{{../Book/figures/}}

\title{Mujeres en las disciplinas STEM}
\date{3 de Febrero de 2021}            
\author{Tecnologías para la Sociedad Digital}

\begin{document}

\input{../Config/setlanguagedependentissues.tex}
\maketitle
\begin{description}                               
  \ifthenelse{\equal{\mybooklanguage}{english}}   
  {        
  \item[Título en inglés:] \mybooktitleenglish   
  }                                            
  {                                           
  }                                              
\item[Equipo:] E6 
\item[Autores:] Carlos Yanguas Durán, Alberto Larena Luengo, David Márquez Mínguez                 
\item[\expandafter\makefirstuc\expandafter{\mybookTutorOrTutores}:] Jesús Alpuente Hermosilla, Juan Antonio Martínez Rojas      
\end{description}      
\setcounter{page}{1}


\begin{center}
 {\bfseries \Large Resumen}
\end{center}
La falta de la mujer en una disciplina es negativa para todos, tanto para la mujer, como para la propia disciplina. 
Pero acusar dicha escasez, a una cuestión de discriminación, es un error. Al igual que no existen ningún impedimento 
para el hombre, poder estudiar carreras relacionadas con la salud, tampoco existe ningún impedimento para la mujer
 estudiar carreras relacionadas con la ciencia. La falta de la mujer dentro de la ciencia, las matemáticas o la ingeniería,
 se debe a una cuestión de gustos. Debe ser la propia mujer por convicción propia la que se decida a estudiar carreras
 relacionadas con las tecnologías, no por una cuestión de porcentajes, ni por la presión de su propio colectivo.               

\section{Introducción}
\label{sec:introduccion}

Parece increíble defender la libertad de la mujer de la forma en la que se hace actualmente en la sociedad, 
y al mismo tiempo concienciarlas e incluso obligarlas a estudiar carreras relacionadas con la ciencia o ingeniería,
solo por el hecho de que en dicho ámbito hay una gran presencia masculina. 

Esta forma de pensar es muy propia de los colectivismos en general, es una forma de codificar la realidad, para 
enfrentar a grupos de hombres y grupos de mujeres como si fuesen equipos de futbol. Las posibilidades de hacer 
comparaciones entre diferentes grupos de personas son infinitas, pero da la casualidad de que siempre son entre 
hombres y mujeres. 

Por lo visto, es muy importante que las mujeres dejen de lado sus, gustos, sus deseos e inquietudes para estudiar 
carreras relacionadas con disciplinas STEM, solo para aumentar los porcentajes del colectivo.

\section{Discusión}
\label{sec:discusion}

El articulo acuña a la brecha de genero existente en la sociedad, el hecho de que el número de mujeres dentro de las 
disciplinas STEM sea reducido con respecto al número de hombres. Parece que durante un momento se ha dejado de 
lado la libertad de decisión, el pensamiento propio y el interés de la mujer en el campo científico, para concienciarlas y 
convencerlas de que deben interesarse en otros ámbitos, con el fin de que la mujer como colectivo, cobre más poder.

Este es un pensamiento de colmena, como si el hecho de que hubiese muchos hombres en los puestos de dirección del 
Ibex 35 afectase positivamente, a un hombre que trabaja en una obra, solo por el hecho de pertenecer a su mismo colectivo.

El único motivo por el que tanto mujeres como hombres toman caminos tan separados dentro del aprendizaje se debe 
meramente a una cuestión de gustos. El hombre suele tener una visión mas teórica relacionada con el aprendizaje, por 
eso se sienten más atraídos por carreras del ámbito científico. Ellas buscan la utilidad práctica, quieren que su trabajo 
reporte un beneficio a la sociedad. Esto es una tendencia, no es algo que determine a las mujeres y a los hombres hacia 
un lado o hacia otro, hay mujeres muy buenas en matemáticas, y hombres muy malos en matemáticas, pero como siempre 
sucede, los resultados difieren entre dos grupos que son diferentes.

Parece necesario e incluso imperativo, que la mujer estudie carreras relacionadas con la ciencia o ingeniería. El problema es 
cuando no se hace por el hecho de poder aportar otro punto de vista dentro de los ámbitos, o para que no sean disciplinas 
sesgadas, sino por el hecho de aumentar el poder de la mujer como colectivo dentro de la sociedad. Aumentar el poder de 
la mujer como colectivo dentro de la sociedad, bajo mi punto de vista, no tiene beneficios sobre la mujer como individuo. 
Según esta filosofía, mujeres con trabajos más humildes, tendrían el lujo de pertenecer a un grupo, donde algunas de 
ellas tienen trabajos relacionados con los números, y podrían mirar a la cima orgullosas de saber que otras mujeres, que
 no tiene nada que ver con ellas, trabajan de programadoras en la última actualización del iPhone 11s.

En cuanto a la falta de referentes femeninos dentro del ámbito STEM, es cierto que es algo que debe cambiar, no por el bien 
de la mujer, sino por el bien del propio ámbito, como ya se ha explicado anteriormente. Se acusa el problema, a la discriminación 
por parte del sistema educativo a las chicas de entre cuatro y seis años, algo que no tiene sentido. La supuesta discriminación 
educativa hacia la mujer desaparece al revisar las estadísticas de fracaso escolar, donde en este caso el hombre sale perdiendo.


\section{Conclusiones}
\label{sec:conclusiones}

La idea de una supuesta discriminación por parte del hombre hacia la mujer dentro de la ciencia debe acabar, no tendría 
sentido que el hombre para seguir aumentando su poder social discriminase a otro colectivo, negando un avance no solo 
en el ámbito científico sino en el resto de áreas STEM. 

Aumentar la inclusión de la mujer dentro del ámbito científico, matemático e ingenieril sería un aspecto muy positivo para 
estos, pero debe ser la propia mujer, no como colectivo, sino como individuo, la que deba tomar dicha decisión. Debemos 
alejarnos de comparaciones y polémicas sobre porcentajes, y profundizar más en la raíz del problema. No solo 
debemos quedarnos con la alta presencia masculina frente a la femenina dentro de áreas científicas, se deben analizar los 
motivos por los que la mujer tiene un menor interés en las mismas y actuar en consecuencia. \nocite{maldita}

\input{../Book/biblio/bibliography.tex}          


\end{document}