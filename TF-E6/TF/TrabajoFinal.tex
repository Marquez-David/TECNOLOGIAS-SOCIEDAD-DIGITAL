%archivos de configuracion del trabajo
\input{../Config/preamble.tex} 
\input{../Config/myconfig.tex}  
\input{../Config/postamble.tex}  

% directorios donde se guardaran las imagenes, diagramas, figuras...
\graphicspath{{../Book/figures/}{../Book/diagrams/}{../Book/photos/}} 

\begin{document} 
% "portada" inicial
\title{{\Huge Ciberseguridad: \\ Amenazas y vulneravilidades en la red}}
\date{{\large 31 de Diciembre de 2020}}
\author{{\Large Tecnologías para la Sociedad Digital}}

%comienza el trabajo
%he puesto los contenidos minimos que hay que hacer, podemos annadir mas secciones


\input{../Config/setlanguagedependentissues.tex}

\maketitle
\begin{description}                                                      
\item [Equipo:] Equipo de trabajo número 6  
\item [Autores:] Carlos Yanguas Durán, Alberto Larena Luengo, David Márquez Mínguez   
\item[Tutor:] Jesús Alpuente Hermosilla      
\end{description}  


%-------------------------------------------RESUMEN------------------------------------------------------------------------------------
\begin{center}
\section*{Resumen}
\label{sec:resumen}
\addcontentsline{toc}{chapter}{Resumen}
\end{center}

Aqui va el resumen


                  



%-------------------------------------------INDICE: SE VA HACIENDO SOLO------------------------------------------------------------------------------------
\input{indice.tex}
\hypersetup{linkcolor=blue}

                        
%-------------------------------------------PARTE 1------------------------------------------------------------------------------------

\chapter{Introducción}
\label{cha:introduccion}


Esddvdfvdf \cite{ejemplo} edefvd Holaghnghng holaá.

Este texto estara sangrado.

\begin{figure}[tphb]
  \centering
  \includegraphics[width=4in]{ejemplo-imagen.jpg}
  \caption{Ejemplo imagen.}
  \label{img:imagen-ejemplo}
\end{figure}

Referencio la imagen \ref{img:imagen-ejemplo}, ahora referencio un capitulo \ref{cha:tipos-sistemas}


%-------------------------------------------PARTE 2------------------------------------------------------------------------------------
\chapter{Vulnerabilidades y ataques en los sistemas informáticos}
\label{cha:vulneravilidades-y-ataques}

\section{Incluyo una nueva seccion}
\label{sec:prueba}
sdfsdfgsdpruebaaa



%-------------------------------------------PARTE 3------------------------------------------------------------------------------------
\chapter{Sistemas de riesgo}
\label{cha:sistemas-de-riesgo}




%-------------------------------------------PARTE 4-----------------------------------------------------------------------------------
\chapter{Ciberataques y sus consecuencias}
\label{cha:ciberataques-y-consecuencias}



%-------------------------------------------PARTE 5------------------------------------------------------------------------------------
\chapter{Planificación de respuesta a incidentes}
\label{cha:planificacion-de-respuesta}



%-------------------------------------------PARTE 6------------------------------------------------------------------------------------
\chapter{Protección informática}
\label{cha:proteccion-informatica}



%-------------------------------------------PARTE 6------------------------------------------------------------------------------------
\chapter{Sistemas utilizados para mantener la seguridad y la privacidad}
\label{cha:tipos-sistemas}

En primer lugar.\cite{rstudio}. holaaa






% fin del trabajo
%se incluye el archivo de biobliografia
\input{../Book/biblio/bibliography.tex}

\end{document}

